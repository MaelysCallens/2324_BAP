%%=============================================================================
%% Inleiding
%%=============================================================================

\chapter{\IfLanguageName{dutch}{Inleiding}{Introduction}}%
\label{ch:inleiding}

In een tijdperk waarin data als het nieuwe goud beschouwd wordt, staat het optimaliseren van de gegevenskwaliteit centraal in het streven naar een succesvolle bedrijfsvoering. Door de exponentiële groei van informatie binnenin organisaties, voornamelijk door Enterprise Resource Planning (ERP) systemen, wordt de essentiële rol van het beheren en het verrijken van master data benadrukt. Deze gegevens vormen de ruggengraat van de operationele processen, de besluitvorming en de klantinteracties. Hierdoor wordt hun accuraatheid en consistentie van vitaal belang voor organisatorische efficiëntie en concurrentievoordeel.
\\
Echter, in de bedrijfsomgevingen blijft het beheer van master data een complex en uitdagend vraagstuk. Het gebrek aan uniformiteit, de aanwezigheid van duplicaten en de inconsistente gegevensformaten vormen obstakels voor een effectief gegevensbeheer. Dit leidt dan tot fouten, inefficiënties en gemiste kansen voor bedrijven om waarde uit hun gegevens te halen. 
\\
Hierbij is de rol van Machine Learning (ML) en Artificial Intelligence (AI) steeds belangrijker geworden. Deze technologieën bieden krachtige instrumenten om de kwaliteit en de bruikbaarheid van gegevens te verbeteren. Door het vermogen om patronen te identificeren, trends te voorspellen en complexe taken uit te voeren, bieden ML en AI een veelbelovend perspectief voor het verhogen van de waarde die uit de data gehaald kan worden.
\\ \\
De kernvraag die in deze bachelorproef wordt bestudeerd, is hoe Machine Learning en Artificial Intelligence kunnen worden ingezet om de kwaliteit van artikel data te verhogen. Deze vraag vormt de basis voor het onderzoek dat gericht is op het identificeren van methoden en technieken die organisaties kunnen toepassen om de kwaliteit van artikel data te verbeteren door gebruik te maken van ML en AI.
\\ \\
Het beoogde resultaat van deze bachelorproef is een proof-of-concept waarbij er een model gecreëerd wordt dat helpt bij het verbeteren van de kwaliteit van artikel data. Dit proof-of-concept zal worden ontwikkeld met als doel het identificeren van duplicaten vooraleer de gebruiker de nieuwe data kan toevoegen in het systeem. Het model zal aangeven wanneer er een duplicaat aanwezig is in het systeem, waardoor het proces voor gebruikers wordt gestroomlijnd en de kans op fouten wordt verminderd.
\\
Bovendien zal het model continu getraind worden om zijn nauwkeurigheid en effectiviteit te verbeteren naarmate het meer data verwerkt en zo nieuwe patronen kan leren herkennen. Door middel van dit proof-of-concept wordt er bijgedragen aan de optimalisatie van gegevensbeheer binnen moderne bedrijfsomgevingen en zo organisaties in staat stellen om beter gebruik te maken van hun waardevolle bedrijfsgegevens.

\section{\IfLanguageName{dutch}{Opzet van deze bachelorproef}{Structure of this bachelor thesis}}%
\label{sec:opzet-bachelorproef}

% Het is gebruikelijk aan het einde van de inleiding een overzicht te
% geven van de opbouw van de rest van de tekst. Deze sectie bevat al een aanzet
% die je kan aanvullen/aanpassen in functie van je eigen tekst.

De rest van deze bachelorproef is als volgt opgebouwd:

In Hoofdstuk~\ref{ch:stand-van-zaken} wordt een overzicht gegeven van de stand van zaken binnen het onderzoeksdomein, op basis van een literatuurstudie.

In Hoofdstuk~\ref{ch:methodologie} wordt de methodologie toegelicht en worden de gebruikte onderzoekstechnieken besproken om een antwoord te kunnen formuleren op de onderzoeksvragen.

% TODO: Vul hier aan voor je eigen hoofstukken, één of twee zinnen per hoofdstuk
In Hoofdstuk~\ref{ch:Onderzoek} wordt het onderzoek besproken en worden de requirements voor het model geïdentificeerd.

In Hoofdstuk~\ref{ch:ProofOfConcept} wordt de proof-of-concept besproken en wordt de werking van het model toegelicht.

%In Hoofdstuk~\ref{ch:IntegratieSAP} wordt de integratie van het model in SAP Master Data Governance besproken en worden de technieken en methoden voor deze integratie geïdentificeerd.

In Hoofdstuk~\ref{ch:conclusie}, tenslotte, wordt de conclusie gegeven en een antwoord geformuleerd op de onderzoeksvragen. Daarbij wordt ook een aanzet gegeven voor toekomstig onderzoek binnen dit domein.