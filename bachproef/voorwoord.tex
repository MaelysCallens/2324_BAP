%%=============================================================================
%% Voorwoord
%%=============================================================================

\chapter*{\IfLanguageName{dutch}{Woord vooraf}{Preface}}%
\label{ch:voorwoord}

%% TODO:
%% Het voorwoord is het enige deel van de bachelorproef waar je vanuit je
%% eigen standpunt (``ik-vorm'') mag schrijven. Je kan hier bv. motiveren
%% waarom jij het onderwerp wil bespreken.
%% Vergeet ook niet te bedanken wie je geholpen/gesteund/... heeft

Deze bachelorproef wordt geschreven in het kader voor het voltooien van de opleiding Toegepaste Informatica. Als student van deze opleiding met een afstudeerrichting in Functional en Business Analyst, ben ik gedurende mijn opleiding gefascineerd geraakt door de mogelijkheden van technologie om problemen op te lossen en processen te verbeteren.
\\ \\
Deze bachelorproef is het resultaat van mijn nieuwsgierigheid en toewijding gedurende mijn laatste jaar van mijn bacheloropleiding. Het onderzoek, de analyse en de proof-of-concept die ik heb uitgevoerd, hebben tot doel om te verkennen hoe geavanceerde technologieën kunnen worden toegepast om de kwaliteit van artikeldata te verbeteren.
\\ \\
Ik wil graag mijn oprechte dank uitspreken aan mijn promotor, Sebastiaan Labijn, en mijn co-promotor, Nicholas Vermeersch, voor hun waardevolle begeleiding, inzichten en ondersteuning gedurende dit proces. Hun expertise en feedback hebben me geholpen om mijn ideeën te verfijnen en mijn onderzoek naar nieuwe hoogten te brengen.
\\ \\
Ik wil ook mijn dankbaarheid uitspreken aan mijn medestudenten, vrienden en familie voor hun voortdurende steun en aanmoediging. Hun geloof in mij heeft me gemotiveerd om door te zetten, zelfs in uitdagende tijden.
\\ \\ % \\ = \newline = nieuwe lijn
Ik wens u veel leesplezier,
\newline Maëlys Callens
\newline Bachelor Toegepaste Informatica, afstudeerrichting functional en business analyst
\newline Hogeschool Gent
