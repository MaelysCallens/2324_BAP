%%=============================================================================
%% Conclusie
%%=============================================================================

\chapter{Conclusie}%
\label{ch:conclusie}

% TODO: Trek een duidelijke conclusie, in de vorm van een antwoord op de
% onderzoeksvra(a)g(en). Wat was jouw bijdrage aan het onderzoeksdomein en
% hoe biedt dit meerwaarde aan het vakgebied/doelgroep? 
% Reflecteer kritisch over het resultaat. In Engelse teksten wordt deze sectie
% ``Discussion'' genoemd. Had je deze uitkomst verwacht? Zijn er zaken die nog
% niet duidelijk zijn?
% Heeft het onderzoek geleid tot nieuwe vragen die uitnodigen tot verder 
%onderzoek?

Dit onderzoek heeft de onderzoeksvraag “Hoe kan Machine Learning/Artificial Intelligence gebruikt worden om de kwaliteit van artikel data te verhogen?” behandeld. 
\\ \\Tijdens deze studie zijn er verschillende requirements opgesteld door middel van een requirementsanalyse. Door deze requirments op te stellen kon er onderzoek verricht worden naar allerlei tools die instaan voor duplicaat detectie. Deze tools werden beoordeeld op basis van deze requirments. De eerste requirements is of de tool duplicaten kan achterhalen op basis van schrijffouten of variaties op schrijfwijzen. Gebruikers kunnen bijvoorbeeld een aantal artikels aanmaken die tot hetzelfde artikel behoren, maar allemaal anders geschreven zijn. De tweede requirment is de herkenning van fonetische woorden. Vaak sluipen er fouten in de dataset omdat men de artikel naam schrijft zoals men het hoort en deze zouden dan zo achterhaalt kunnen worden. De derde requirment is of duplicaten herkend kunnen worden door middel van standaardisatie. Zo kunnen er verschillende afkortingen voorkomen in de naam van een artikel, maar soms kunnen deze afkortingen ook voluit geschreven worden. Door standaardisatie toe te passen kunnen duplicaten hierop gefilterd worden. De laatste requirement is het nagaan of een tool kan omgaan met speciale karakters en tekens. Hierdoor kunnen ook al veel fouten opgespoord worden.
\\ \\Uit de studie blijkt dat Artificiële Intelligentie en Machine Learning significant kunnen bijdragen aan de verbetering  van de kwaliteit van artikel data. Hoewel de proof-of-concept enkele tekortkomingen vertoonde, zijn er tools zoals Synity Match die met behulp van Machine Learning krachtige oplossingen voor duplicaat detectie bieden. Deze conclusie stemt overeen met het op voorhand beoogde resultaat van het onderzoek. Bedrijven die streven naar hogere en betere datakwaliteit kunnen door deze bevindingen worden aangemoedigd om machine learning-modellen te implementeren.
\\ \\De resultaten van dit onderzoek kunnen bovendien dienen als een stimulans voor verdere verkenning en verbetering van de huidige algoritmen die worden ingezet hiervoor. De voordelen van kunstmatige intelligentie in dit domein zijn duidelijk aangetoond, wat de optimalisatie van deze technologieën en hun naadloze integratie met master data management systemen, zoals SPA Master Data Governance, essentieel maakt.
