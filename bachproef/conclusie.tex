%%=============================================================================
%% Conclusie
%%=============================================================================

\chapter{Conclusie}%
\label{ch:conclusie}

% TODO: Trek een duidelijke conclusie, in de vorm van een antwoord op de
% onderzoeksvra(a)g(en). Wat was jouw bijdrage aan het onderzoeksdomein en
% hoe biedt dit meerwaarde aan het vakgebied/doelgroep? 
% Reflecteer kritisch over het resultaat. In Engelse teksten wordt deze sectie
% ``Discussion'' genoemd. Had je deze uitkomst verwacht? Zijn er zaken die nog
% niet duidelijk zijn?
% Heeft het onderzoek geleid tot nieuwe vragen die uitnodigen tot verder 
%onderzoek?

Dit onderzoek heeft de onderzoeksvraag “Hoe kan Machine Learning/Artificial Intelligence gebruikt worden om de kwaliteit van artikel data te verhogen?” behandeld. 
\\
...
\\
Deze bachelorproef leidt tot de conclusie dat Artificiële Intelligentie en Machine Learning daadwerkelijk een aanzienlijke bijdrage kunnen leveren bij het verhogen van de kwaliteit van artikel data. … Deze conclusie komt overeen met het op voorhand beoogde resultaat van het onderzoek. Het kan bedrijven die gedreven zijn om data quality te verhogen en verzekeren helpen om de keuze te maken voor een machine learning model.
\\
De bevindingen van dit onderzoek kunnen daarbovenop als een stimulans dienen voor verdere verkenning en verbetering van de huidige algoritmen die worden ingezet hiervoor . De evidentie van het nut van kunstmatige intelligentie in dit domein is na dit onderzoek onmiskenbaar, waardoor de optimalisatie van deze technologieën en een naadloze integratie met master data management systemen, zoals SAP Master Data Governance, van cruciaal belang zijn voor bedrijven zoals Alluvion die streven naar kwaliteitsvolle en betrouwbare data voor hun klanten. 