%%=============================================================================
%% Conclusie
%%=============================================================================

\chapter{Conclusie}%
\label{ch:conclusie}

% TODO: Trek een duidelijke conclusie, in de vorm van een antwoord op de
% onderzoeksvra(a)g(en). Wat was jouw bijdrage aan het onderzoeksdomein en
% hoe biedt dit meerwaarde aan het vakgebied/doelgroep? 
% Reflecteer kritisch over het resultaat. In Engelse teksten wordt deze sectie
% ``Discussion'' genoemd. Had je deze uitkomst verwacht? Zijn er zaken die nog
% niet duidelijk zijn?
% Heeft het onderzoek geleid tot nieuwe vragen die uitnodigen tot verder 
%onderzoek?

Dit onderzoek heeft de onderzoeksvraag “Hoe kan Machine Learning/Artificial Intelligence gebruikt worden om de kwaliteit van artikel data te verhogen?” behandeld. 
\\ \\Tijdens deze studie zijn er verschillende requirements opgesteld door middel van een requirementsanalyse. Door deze requirements op te stellen kon er onderzoek verricht worden naar allerlei tools die instaan voor duplicaat detectie. Deze tools werden beoordeeld op basis van deze requirements. De eerste requirements is of de tool duplicaten kan achterhalen op basis van schrijffouten of variaties op schrijfwijzen. Gebruikers kunnen bijvoorbeeld een aantal artikels records aanmaken die tot hetzelfde artikel behoren, maar allemaal anders geschreven zijn. De tweede requirement is de herkenning van fonetische woorden. Vaak sluipen er fouten in de dataset omdat men de artikel naam schrijft zoals men het hoort en door de herkenning hiervan kunnen deze fouten ontdekt worden. De derde requirement is of duplicaten herkend kunnen worden door middel van standaardisatie. Zo kunnen er verschillende afkortingen voorkomen in de naam van een artikel, maar in sommige records kunnen deze afkortingen voluit geschreven worden. Door standaardisatie toe te passen kunnen duplicaten hierop gefilterd worden. De laatste requirement is het nagaan of een tool kan omgaan met speciale karakters en tekens. Hierdoor kunnen ook al veel fouten opgespoord worden, wanneer de artikels hiervan gebruik maken.
\\De tools die in deze bachelorproef besproken zijn, zijn SAP, SAP Master Data Governance, Syniti Match, IBM Info Sphere Master Data Management, Pimcore Product Information Management en Pimcore Customer Management Framework.  SAP biedt niet standaard een geavanceerde duplicaat detectie aan, wanneer men de naam van het artikel intypt kan SAP niet herkennen of dit artikel al aanwezig is in het systeem, ook al verschilt het maar met één karakter. Hierdoor wordt er bij SAP aan geen enkele requirement voldaan. Doordat SAP met dit probleem kaapt, is er een aanvulling ontworpen, namelijk SAP MDG, die dit probleem oplost. SAP MDG zorgt voor een uniforme en consistente weergave van de master data. SAP MDG maakt het mogelijk om aan fuzzy search te doen.  Hierdoor wordt er enkel aan de eerste requirement voldaan. Bij Syniti Match, wordt er automatische extractie, normalisatie en parsing toegepast. Synity Match is een volwaardig product, die aan alle vier de requirements voldoet. IBM InfoSphere Master Data Management maakt gebruik van probalistic matching en deterministic matching.  IBM InfoSphere Master Data Management voldoet enkel aan de eerste requirement. Als laatste zijn Pimcore Product Information Management en Pimcore Customer Management Framework in dit onderzoek besproken geweest. Het Customer Management Framework van Pimcore bevat een Customer Duplicates Service. Het is een duplicaatdetectiesysteem die het mogelijk maakt om duplicaten op te sporen binnen klantgegevens. Het framework van Pimcore voldoet aan de eerste en tweede requirement omdat het gebruik maakt van soundex- en metaphone- algoritme.
\\ \\Uit de studie blijkt dat Artificiële Intelligentie en Machine Learning significant kunnen bijdragen aan de verbetering  van de kwaliteit van artikel data. Hoewel de proof-of-concept enkele tekortkomingen vertoonde, zijn er tools zoals Synity Match die met behulp van Machine Learning krachtige oplossingen voor duplicaat detectie bieden. Deze conclusie stemt overeen met het op voorhand beoogde resultaat van het onderzoek. Bedrijven die streven naar hogere en betere datakwaliteit kunnen door deze bevindingen worden aangemoedigd om machine learning-modellen te implementeren.
\\ \\De resultaten van dit onderzoek kunnen bovendien dienen als een stimulans voor verdere verkenning en verbetering van de huidige algoritmen die worden ingezet hiervoor. De voordelen van kunstmatige intelligentie in dit domein zijn duidelijk aangetoond, wat de optimalisatie van deze technologieën en hun naadloze integratie met master data management systemen, zoals SAP Master Data Governance, essentieel maakt.
