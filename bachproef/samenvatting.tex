%%=============================================================================
%% Samenvatting
%%=============================================================================

% TODO: De "abstract" of samenvatting is een kernachtige (~ 1 blz. voor een
% thesis) synthese van het document.
%
% Een goede abstract biedt een kernachtig antwoord op volgende vragen:
%
% 1. Waarover gaat de bachelorproef?
% 2. Waarom heb je er over geschreven?
% 3. Hoe heb je het onderzoek uitgevoerd?
% 4. Wat waren de resultaten? Wat blijkt uit je onderzoek?
% 5. Wat betekenen je resultaten? Wat is de relevantie voor het werkveld?
%
% Daarom bestaat een abstract uit volgende componenten:
%
% - inleiding + kaderen thema
% - probleemstelling
% - (centrale) onderzoeksvraag
% - onderzoeksdoelstelling
% - methodologie
% - resultaten (beperk tot de belangrijkste, relevant voor de onderzoeksvraag)
% - conclusies, aanbevelingen, beperkingen
%
% LET OP! Een samenvatting is GEEN voorwoord!

%%---------- Nederlandse samenvatting -----------------------------------------
%
% TODO: Als je je bachelorproef in het Engels schrijft, moet je eerst een
% Nederlandse samenvatting invoegen. Haal daarvoor onderstaande code uit
% commentaar.
% Wie zijn bachelorproef in het Nederlands schrijft, kan dit negeren, de inhoud
% wordt niet in het document ingevoegd.

\IfLanguageName{english}{%
\selectlanguage{dutch}
\chapter*{Samenvatting}
\lipsum[1-4]
\selectlanguage{english}
}{}

%%---------- Samenvatting -----------------------------------------------------
% De samenvatting in de hoofdtaal van het document

\chapter*{\IfLanguageName{dutch}{Samenvatting}{Abstract}}

In een tijd waarin bedrijven steeds meer data genereren en gebruiken, is het verbeteren van de kwaliteit van gegevens essentieel geworden voor een succesvolle bedrijfsvoering. Dit geldt met name voor master data, die fungeren als de kern van operationele processen, besluitvorming en klantinteracties. Ondanks het belang ervan blijft het beheer van master data een uitdagend vraagstuk vanwege problemen zoals inconsistentie en duplicatie. Machine Learning (ML) en Artificial Intelligence (AI) worden steeds belangrijker omdat ze krachtige instrumenten bieden om de kwaliteit en bruikbaarheid van gegevens te verbeteren.
\\
Het doel van dit onderzoek is om methoden en technieken te identificeren die organisaties kunnen toepassen om de kwaliteit van artikel data te verbeteren door ML en AI te benutten. Een proof-of-concept wordt ontwikkeld om een model te creëren dat de kwaliteit van artikel data kan verbeteren door een duplicaat detectie systeem op te stellen, waardoor het gebruikersproces wordt gestroomlijnd en de kans op fouten wordt verminderd. 
\\
Het onderzoek draagt bij aan de optimalisatie van gegevensbeheer binnen moderne bedrijfsomgevingen, waardoor organisaties beter gebruik kunnen maken van hun waardevolle bedrijfsgegevens.

