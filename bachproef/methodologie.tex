%%=============================================================================
%% Methodologie
%%=============================================================================

\chapter{\IfLanguageName{dutch}{Methodologie}{Methodology}}%
\label{ch:methodologie}

Het onderzoek naar het gebruik van Machine Learning (ML) en Artificial Intelligence (AI) om de kwaliteit van artikel data te verbeteren, omvat een uitgebreid plan van aanpak dat verschillende fasen omvat.

\section{\IfLanguageName{dutch}{Literatuurstudie}{Literature study}}%
\label{sec:LiteratuurstudieM}
De eerste fase van het plan van aanpak omvat een diepgaande literatuurstudie om een grondig begrip over het onderwerp te verkrijgen. Er zal gezocht worden naar relevant onderzoek, academische papers en boeken om een inzicht te verkrijgen in Enterprise Resource Planning (ERP), master data, Master Data Management (MDM), Master Data Governance (MDG), Artificial Intelligence (AI) en Machine Learning (ML).

\section{\IfLanguageName{dutch}{Onderzoek}{Research}}%
\label{sec:OnderzoekM}
Na de literatuurstudie zal er een onderzoek uitgevoerd worden naar de bestaande tools en technieken op het gebied van duplicaat detectie en hun functionaliteiten. Eerst zal er een requirementsanalyse opgesteld worden. In deze analyse worden de eisen vastgelegd waarmee de tools worden vergeleken, evenals de criteria waaraan het Proof-of-Concept moet voldoen. Vervolgens zullen alle tools worden vergeleken op basis van hun voor- en nadelen en de gestelde requirements. Uit dit onderzoek zal blijken welke tekortkomingen er zijn binnen het domein van duplicaat detectie en op welke aspecten verbetering mogelijk is.

\section{\IfLanguageName{dutch}{Proof-of-Concept}{Proof-of-Concept}}%
\label{sec:Proof-of-ConceptM}
De derde fase omvat verschillende aspecten, waaronder de bepaling van de doelstelling, de selectie van de techniek en de datasetverzameling. De doelstellingen worden vastgesteld op basis van de bevindingen van het onderzoek. De verschillende technieken en mogelijke oplossingen zullen vergeleken worden met elkaar om het meest geschikte model te kunnen opstellen. Daarnaast zal er een representatieve dataset verzameld worden, deze is van cruciaal belang voor het testen en trainen van het model. 
\\Na de voorbereidende fase wordt het proof-of-concept effectief uitgewerkt. Dit omvat het ontwerpen en implementeren van het model met behulp van ML/AI. Het model zal daarna getraind worden met de verzamelde dataset om duplicaten te kunnen identificeren die aanwezig zijn in het systeem. 

%\section{\IfLanguageName{dutch}{Integratie in SAP Master Data Governance}{Integration in SAP Master Data Governance}}%
%\label{sec:IntegratieM}
%Na het ontwerpen van het model in het proof-of-concept, zal er onderzocht worden hoe deze oplossing kan worden geïntegreerd in de SAP Master Data Governance (MDG) tool. Dit omvat het identificeren van de beste methoden en technieken voor deze integratie, zoals het gebruik van API's.

\section{\IfLanguageName{dutch}{Conclusie}{Conclusion}}%
\label{sec:ConclusieM}
Na de implementatie wordt de effectiviteit en efficiëntie van het proof-of-concept geanalyseerd. Er wordt een conclusie opgesteld die verteld of het ontwerpen en integreren van het model al dan niet een handige tool is binnenin SAP MDG voor het identificeren van duplicaten in artikel data. 
\\ \\
Door deze methodologie te volgen, wordt er een gestructureerde aanpak gehanteerd om het onderzoek uit te voeren en de doelstellingen te bereiken. Dit maakt het mogelijk om de impact van ML/AI op het verbeteren van de kwaliteit van artikel data grondig te onderzoeken en waardevolle inzichten te verkrijgen voor toekomstige toepassingen.