%%=============================================================================
%% Methodologie
%%=============================================================================

\chapter{\IfLanguageName{dutch}{Methodologie}{Methodology}}%
\label{ch:methodologie}

\large MOET NOG VERDER EN BETER UITGESCHREVEN WORDEN

Het onderzoek naar het gebruik van Machine Learning (ML) en Artificial Intelligence (AI) om de kwaliteit van artikel data te verbeteren, omvat een uitgebreid plan van aanpak dat verschillende fasen omvat.

\section{\IfLanguageName{dutch}{Literatuurstudie}{Literatuurstudie}}%
\label{sec:literatuurstudieM}
De eerste fase omvat een diepgaande literatuurstudie om een grondig begrip van het onderwerp te verkrijgen. Relevant onderzoek, academische papers en boeken worden bestudeerd om inzicht te krijgen in master data, technieken om de kwaliteit ervan te verbeteren, en technologieën die ML/AI gebruiken voor dit doel.

\section{\IfLanguageName{dutch}{Proof-of-Concept}{Proof-of-Concept}}%
\label{sec:Proof-of-ConceptM}
Deze fase omvat verschillende aspecten, waaronder de bepaling van de doelstelling, de selectie van de techniek en de datasetverzameling. De doelstellingen worden vastgesteld op basis van de bevindingen uit de literatuurstudie. Hierbij wordt bepaald welke techniek of oplossing het meest geschikt is om de kwaliteit van artikel data te verbeteren. Daarnaast wordt een representatieve dataset verzameld, cruciaal voor het testen en trainen van het model.
Na de voorbereidende fase wordt het proof-of-concept effectief uitgewerkt. Dit omvat het ontwerpen en implementeren van het model met behulp van ML/AI. Het model zal worden getraind met de verzamelde dataset om suggesties te bieden bij het invoeren van nieuwe velden voor artikel data. Het doel is om duplicaten en tegenstrijdigheden te minimaliseren en zo de kwaliteit van de gegevens te verbeteren.

\section{\IfLanguageName{dutch}{Integratie in SAP Master Data Governance (MDG)}{Integration in SAP Master Data Governance (MDG)}}%
\label{sec:IntegratieM}
Na de ontwikkeling van het proof-of-concept wordt onderzocht hoe deze oplossing kan worden geïntegreerd in de SAP Master Data Governance (MDG) tool. Dit omvat het identificeren van de beste methoden en technieken voor integratie, zoals het gebruik van API's.

\section{\IfLanguageName{dutch}{Analyse van resultaten}{Analysis van resultaten}}%
\label{sec:ConclusieM}
Na implementatie wordt de effectiviteit van het proof-of-concept geanalyseerd aan de hand van verzamelde data en gebruikerservaringen. Eventuele verbeteringen worden geïdentificeerd en aanbevelingen worden geformuleerd voor verdere optimalisatie en toepassing in de praktijk.
\\ \\
Door deze methodologie te volgen, wordt een gestructureerde aanpak gehanteerd om het onderzoek uit te voeren en de doelstellingen te bereiken. Dit stelt ons in staat om de impact van ML/AI op het verbeteren van de kwaliteit van artikel data grondig te onderzoeken en waardevolle inzichten te verkrijgen voor toekomstige toepassingen.