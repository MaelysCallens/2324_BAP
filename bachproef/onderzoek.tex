%%=============================================================================
%% Onderzoek
%%=============================================================================

\chapter{\IfLanguageName{dutch}{Onderzoek}{Research}}%
\label{ch:Onderzoek}

Zoals beschreven in Hoofdstuk~\ref{ch:methodologie}, zal er een onderzoek gedaan worden naar de requirements waaraan een duplicaatdetectiesysteem moet voldoen. Daarna zal er gekeken worden naar bestaande tools en technieken die al aanwezig zijn op de markt op vlak van duplicaten en hun functionaliteiten. 

\section{\IfLanguageName{dutch}{Requirementsanalyse}{Requirementsanalyse}}%
\label{sec:requirementsanalyse}

Een duplicaatdetectiesysteem voor SAP artikel data moet aan verschillende eisen voldoen om effectief en efficient te zijn.
\\Een eerste requirement waarop tools kunnen beoordeeld worden is hoe ze omgaan met simpele schrijffouten of variaties van de schrijfwijze van artikelnamen en hun beschrijvingen. Een voorbeeld van zo een schrijffout of variatie is een situatie waarin het bedrijf beschikt over het artikel “PlayStation 5” in hun systeem, maar een andere gebruiker probeert het artikel “Playstation V”, “PS 5” of “Playstation 5”  toe te voegen.
\\Een tweede belangrijk aspect bij een duplicaatdetectiesysteem is of het systeem woorden die fonetisch hetzelfde klinken kan herkennen. Zo hebben de Engelse woorden “queue” en “cue” een andere schrijfwijze, maar de betekenis en uitspraak van deze woorden is wel hetzelfde. Als er verschillende gebruikers zo werken, zullen er veel dezelfde artikels in het systeem aanwezig zijn die op het eerste zicht niet op elkaar lijken.
\\Een derde requirement is in hoeverre tools voldoen aan standaardisatie. Hierbij kan er gekeken worden naar de afkortingen die voor komen in de naam van een artikel. Zo kan een duplicaatdetectiesysteem in een ideaal scenario herkennen dat “Samsung TV” en “Samsung Televisie” hetzelfde betekenen. Verschillende talen en landen hebben elk hun eigen afkortingen, een goede tool kan al deze zaken opsporen.
\\Als laatste requirement is het belangrijk of er onderzocht wordt of een tool goed overweg kan met speciale tekens en karakters. Namen van artikels die in het systeem zitten onder een andere taal, kunnen speciale karakters of tekens bevatten. Zo kan het voorvallen dat “Nintendo – Switch” zich onder de Japanse naam bevindt in het systeem. Duplicaatdetectiesystemen die duplicaten tussen verschillende alfabetten en tekens kunnen herkennen, zijn daarom ook een grote meerwaarde. 
\\In alle gevallen wordt er verwacht dat een duplicaatdetectiesysteem zal aantonen dat er hoge gelijkenissen zijn tussen deze artikelnamen. Idealiter krijgt de gebruiker dan een melding waardoor hij dan zelf kan controleren of het al dan niet om een duplicaat gaat.

\section{\IfLanguageName{dutch}{Tools}{Tools}}%
\label{sec:tools}
Dit hoofdstuk richt zich op het beoordelen van enkele tools die de dag van vandaag op de markt aanwezig zijn op basis van de requirements die in het vorige hoofdstuk zijn opgesteld. De vier vooropgestelde requirements kunnen aftoetsen of een duplicaat detectie tool al dan niet effectief en efficient is.

\subsection{SAP}
Standaard werkt SAP zonder een geavanceerde duplicaat detectie. Bij het aanmaken en opzoeken van een materiaal kan de gebruiker de naam invoeren en wanneer deze naam met maar één karakter verschilt van een bestaande, zal SAP deze naam niet herkennen. Hierdoor kunnen duplicaten makkelijk aangemaakt worden. Met andere woorden onze eerste requirement, het herkennen van schrijffouten en schrijfwijzes, wordt op geen enkele manier herkent binnen een standaard SAP-systeem.
\\Ook wordt standaardisatie niet toegepast. Gebruikers kunnen allerlei afkortingen invullen zonder dat SAP ze herkent als duplicaat. 
\\Een standaard SAP-implementatie voldoet niet aan de andere twee requirements. Het detecteren van duplicaten wordt helemaal niet toegepast binnenin een SAP-systeem. Dit kan voor veel problemen zorgen wanneer verschillende gebruikers in contact komen met artikel data. Er is technisch gezien niets dat het systeem beschermt tegen onvermijdelijke menselijke fouten.

\subsection{SAP Master Data Governance}
SAP biedt nu zelf een tool aan die deze problemen aanpakt. 
\\SAP Master Data Governance (MDG) is een aanvulling die ontworpen is voor het beheer van master data in SAP, zoals eerder vermeld in Hoofdstuk~\ref{ch:stand-van-zaken} \autocite{SAPMDG}. Het zorgt voor een uniforme en consistente weergave van de master data, de workflows en de beleidsregels om de gegevenskwaliteit en-bescherming te waarborgen. 
\\Door middel van MDG kan een gebruiker niet alleen de naam, maar ook andere attributen van een artikel opzoeken. Zo kan er gecontroleerd worden of een bepaalde naam van het artikel al aanwezig is in het systeem. Door deze zoekfunctionaliteit is er een mogelijkheid tot fuzzy search. Hierbij worden schrijffouten en andere variaties in schrijfwijzen herkent als mogelijke duplicaten. 
\\Fuzzy search is een techniek waarbij zoekalgoritmen worden gebuikt om tekenreeksen te vinden die ongeveer overeenkomen met andere patronen \autocite{TechTarget2022}. Fuzzy-zoekopdrachten worden gebruikt voor zoekopdrachten in Structured Query Language (SQL) om databasegebruikers te kunnen helpen records te vinden zonder zeker te zijn van de exacte spelling van de waarde waarnaar ze zoeken. Het wordt gebruikt met behulp van een fuzzy matching-algoritme, dat een lijst met resultaten retourneert op basis van waarschijnlijke relevantie, ook al komen de woorden en de spellingen niet exact overeen met elkaar. Het algoritme kan woorden retourneren die het opgegeven basiswoord bevat samen met voor- en achtervoegsels. Het algoritme vergelijkt twee strings met elkaar en kent aan elke string een score toe op basis van hun overeenkomst. Hoe dichter de scores bij elkaar liggen, hoe meer strings op elkaar lijken. De Fuzzy matching operator berekent de Levenshtein-distance tussen een document en de query. Deze afstandmeting bepaalt de score door de Levenshtein-distances van verschillende documenten met dezelfde query te vergelijken. Daarna wordt er een score tussen 0 en 1 toegekend, waarbij 1 staat voor een exacte overeenkomst. In de praktijk betekent dit dat MDG alleen BTW-nummers van leveranciers matcht als ze 100\% overeenkomen met elkaar. Terwijl een artikelnaamveld als match wordt beschouwt bij een overeenkomt van 75\% of meer. Stel dat de gebruiker op zoek is naar 'PlayStadion 5' en het systeem beschikt over de artikel naam 'PlayStation 5'. Dan zal het systeem een fuzzy matching toepassen door de Levenshtein-distance te berekenen. Aangezien 11 van de 12 letters overeenkomst, is de overeenkomst 91,67\% \autocite{IBM}. Dit is hoger dan de 75\% threshold, dus zal het systeem 'PlayStation 5' als match tonen.
\\Daarnaast biedt SAP MDG een duplicate check functie aan. Hierbij controleert de functie automatisch of de gegevens van een nieuw artikel overeenkomen met een bestaand artikel. De gebruiker krijgt dan een melding als er een mogelijk duplicaat wordt gevonden. 
\\ SAP MDG biedt ook mogelijkheden om het matching proces verder te configueren. Gebruikers kunnen bijvoorbeeld gewichten toekennen aan bepaalde criteria, zoals leveranciers, om de relevatie van overeenkomsten te bepalen.
\\Echter, komen er extra kosten en aanzienlijke inspanningen voor deze implementatie en configuratie boven op de bestaande SAP-infrastructuur en licentiekosten voor het herkennen van schrijffouten binnen SAP MDG.

\subsection{Syniti Match}
\textcite{SynitiAbout} is een bedrijf dat ondernemingen helpt met het leveren van gegevens waarop hun klanten vertrouwen om groei te stimuleren en risico's te verminderen.
\\Ze bieden zowel de technologie, als de expertise die nodig is om een datatraject te ondersteunen, bedrijfskennis te vergroten, nieuwe bronnen van potentieel te ontsluiten en nieuwe mogelijkheden voor groei vrij te maken.
\\Syniti heeft een Data Matching Software, genaamd Syniti Match, waarin duplicaten in data gedetecteerd kunnen worden \autocite{SynitiMatch}. Dit proces gebeurt in verschillende stappen. 
\\Als eerste wordt kwalitatieve analyselogica gebruikt. Hierbij onderzoekt hun Artificiële Intelligentie eerst de data om patronen te herkennen en associaties tussen woorden te maken. Ook wordt onvolledige of incorrecte informatie geïdentificeerd. 
\\Vervolgens wordt de data verwerkt door allerlei verschillende matching-algoritmen. Deze algoritmen worden op een intelligente wijze gecombineerd met elkaar om verschillende soorten gelijkenissen te herkennen. Syniti Match past automatische extractie, normalisatie en parsing toe. De data worden gegroepeerd in clusters en geanalyseerd op gelijkenissen. Dit proces kan meerdere keren herhaald worden, waarbij telkens verschillende groepen en tokens worden gebruikt om verschillende soorten gelijkenissen te detecteren. 
\\Na deze fase worden alle records per groep vergeleken. De verschillende attributen krijgen scores op basis van hun gelijkenis, waarbij elk attribuut een gewicht toegewezen krijgt. Dit gewicht geeft de invloed op de gelijkenis aan. Deze scores worden dan opgeteld om een gelijkheidsscrore te berekenen. Hoe hoger deze score, hoe groter de kans is dat deze twee datarecords over hetzelfde artikel gaan. 
\\Syniti Match lijkt een volwaardig product te zijn die voldoet aan alle vooropgestelde requirements uit het onderzoek. Het kan schrijffouten en woorden identificeren die fonetisch op elkaar lijken. Het maakt ook gebruik van een standaardisatie waarbij verschillende afgekorte woorden kunnen gelinkt worden aan de voluit geschreven versie ervan en tot slot kan Syniti Match ook werken met verschillende speciale tekens en alfabetten om mogelijkse duplicaten te herkennen.

\subsection{IBM InfoSphere Master Data Management}
Als voorlaatste tool wordt IBM Infosphere Master Data Management onderzocht. 
\textcite{IBMAbout}, staat voor International Business Machines Corporation, is een wereldwijd toonaangevend technologiebedrijf dat zich richt op het aanbieden van hardware, software en services op het gebied van informatietechnologie (IT) en bedrijfsoplossingen.
\\Als eerste bestaat er probalistic matching.  Hierbij wordt de waarschijnlijkheid dat een bepaald veld een bepaalde waarde heeft in overweging genomen \autocite{IBMMatching}.
\\Daarnaast bestaat er deterministic matching, hierbij wordt er bij de matching rekening gehouden met de matchende data en het eventuele ontbreken van kritieke data in één of beide records voor het vaststellen van een match. 
\\IBM Infosphere MDM voldoet echter niet aan alle requirements. Hoewel de schrijffouten in data worden herkent, wordt de derde requirement, standaardisatie, niet toegepast. Ook aan de tweede requirement wordt er niet voldaan. Er kunnen geen fonetische matches verkregen worden en ook met de laatste requirement, in verband met de speciale tekens of karakters, wordt er geen rekening gehouden.

\subsection{Pimcore Product Information Management}
De laatste tool die besproken wordt in dit onderzoek is Pimcore Product Information Management (PIM). 
\\ \textcite{Pimcore} is een open source digitaal platform dat gebruikt kan worden om master data te beheren. Pimcore biedt zowel Product Information Management, Master Data Management, Digital Asset Management, Customer Data Platform, Digital Experience Platform and Digital Commerce aan. Ze proberen een betrouwbaar zicht op bedrijfsdata te bieden om zo de operationele efficiëntie te verhogen en de IT kosten te verlagen. Ze proberen zichzelf in de markt te differentiëren als een flexibel en API gedreven alternatief. 
\\Pimcore fungeert met zijn PIM/MDM als een centrale hub voor alle productgerelateerde informatie, gaande van legal informatie over sales- en marketinginformatie tot technische specificaties \autocite{StudioEmma}. Dit voorkomt dat data verspreid raakt over verschillende afdelingen en systemen, wat kan leiden tot inconsistente gegevens.
\\Het Customer Management Framework van Pimcore bevat een Customer Duplicates Service \autocite{Pimcore2024}. Dit is een service waarmee duplicaten binnen klantgegevens kunnen gedetecteerd worden. Dit is dus geen duplicaatdetectiesysteem voor artikel data. Deze service maakt het mogelijk om zowel duplicaten van een specifieke klant te zoeken als om globaal naar alle mogelijke duplicaten binnen een database te zoeken. Het biedt naast traditionele fuzzy matching ook de mogelijkheid om te matchen op basis van het soundex- of metaphone-algoritme. Hierbij kunnen gebruikers alle gevonden potentiële duplicaten bekijken en controleren in de duplicates view. Dan kunnen duplicaten worden samengevoegd tot één record die de beste data bevat.
\\Pimcore voldoet aan de eerste requirements aangezien het schrijffouten kan herkennen bij het zoeken naar duplicaten. Echter, maakt het programma geen gebruik van standaardisatie bij het matchen van records, dus de derde vereiste wordt niet volbracht. In tegendeel tot de andere tools kan er bij Pimcore wel gezocht worden naar woorden die fonetisch op elkaar lijken via het soundex- en metaphone-algoritme. Het gebruik van speciale tekens en karakters wordt niet specifiek behandeld binnen Pimcore.
